\chapter{GUI编程}

\section{GUI编程}

\subsection{GUI编程}

图形用户接口GUI(Graphic User Interface)是人机交互的重要技术手段,利用GUI技术可以方便使用者使用。在不同的编程语言内部实际上也提供有一系列GUI组件。\\

如果要编写出一个图形界面,就必须非常清楚每一种组件的定义及相关的处理操作,同时还需要清除整个界面组件的布局管理。在Python中可以使用tkinter、Pyqt5组件,如果有Java的开发能力,也可以使用Jython通过Java语言类库实现图形化界面开发。在tkinter模块中提供了多种不同的窗体组件:

\begin{table}[H]
	\centering
	\setlength{\tabcolsep}{5mm}{
		\begin{tabular}{|c|l|}
			\hline
			\textbf{组件} & \textbf{描述}                        \\
			\hline
			Button        & 按钮                                 \\
			\hline
			Checkbutton   & 多选框                               \\
			\hline
			Entry         & 输入框                               \\
			\hline
			Frame         & 框架控件,在进行排版时实现子排版模型 \\
			\hline
			Label         & 标签                                 \\
			\hline
			Listbox       & 列表框                               \\
			\hline
			Menu          & 菜单                                 \\
			\hline
			Menubutton    & 菜单按钮,为菜单定义菜单项           \\
			\hline
			Radiobutton   & 单选按钮                             \\
			\hline
			Scale         & 滑动组件                             \\
			\hline
			Scrollbar     & 滚动条组件                           \\
			\hline
			Text          & 文本                                 \\
			\hline
			LabelFrame    & 容器组件,实现复杂组件布局           \\
			\hline
			tkMessageBox  & 消息组件,可以进行提示框的显示       \\
			\hline
		\end{tabular}
	}
	\caption{tkinter模块窗体组件}
\end{table}


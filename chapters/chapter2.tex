\chapter{判断}

\section{逻辑运算符}

\subsection{逻辑运算符}

Python中逻辑运算符有三种:

\begin{enumerate}
	\item 逻辑与and:当多个条件同时为真,结果为真。
	      \begin{table}[H]
		      \centering
		      \setlength{\tabcolsep}{5mm}{
			      \begin{tabular}{|c|c|c|}
				      \hline
				      \textbf{条件1} & \textbf{条件2} & \textbf{条件1 and 条件2} \\
				      \hline
				      T              & T              & T                        \\
				      \hline
				      T              & F              & F                        \\
				      \hline
				      F              & T              & F                        \\
				      \hline
				      F              & F              & F                        \\
				      \hline
			      \end{tabular}
		      }
		      \caption{逻辑与}
	      \end{table}

	\item 逻辑或or:多个条件有一个为真时,结果为真。
	      \begin{table}[H]
		      \centering
		      \setlength{\tabcolsep}{5mm}{
			      \begin{tabular}{|c|c|c|}
				      \hline
				      \textbf{条件1} & \textbf{条件2} & \textbf{条件1 or 条件2} \\
				      \hline
				      T              & T              & T                       \\
				      \hline
				      T              & F              & T                       \\
				      \hline
				      F              & T              & T                       \\
				      \hline
				      F              & F              & F                       \\
				      \hline
			      \end{tabular}
		      }
		      \caption{逻辑或}
	      \end{table}

	\item 逻辑非not:条件为真时,结果为假;条件为假时,结果为真。
	      \begin{table}[H]
		      \centering
		      \setlength{\tabcolsep}{5mm}{
			      \begin{tabular}{|c|c|}
				      \hline
				      \textbf{条件} & \textbf{not 条件} \\
				      \hline
				      T             & F                 \\
				      \hline
				      F             & T                 \\
				      \hline
			      \end{tabular}
		      }
		      \caption{逻辑非}
	      \end{table}
\end{enumerate}

\newpage

\section{if}

\subsection{if}

分支结构最大特征就是可以进行指定条件的判断处理,关键字为if、elif、else。\\

每一个满足条件之后的语句都可以有多条,并且在Python里面是利用缩进来确定语句的关系。使用逻辑运算符可以进行若干个条件的连接。

\subsubsection{单分支}

\vspace{-1cm}

\begin{lstlisting}[language=Python]
age = 15
if 0 < age < 18:
    print("未成年")
\end{lstlisting}

\subsubsection{双分支}

\vspace{-1cm}

\begin{lstlisting}[language=Python]
age = 30
if 0 < age < 18:
    print("未成年人")
else:
    print("成年人")
\end{lstlisting}

\subsubsection{多分支}

\vspace{-1cm}

\begin{lstlisting}[language=Python]
score = 76

if 90 <= score <= 100:
    print("优秀")
elif score >= 60:
    print("合格")
else:
    print("不合格")
\end{lstlisting}

\vspace{0.5cm}

\mybox{判断整数奇偶}

\begin{lstlisting}[language=Python]
num = int(input("输入一个正整数:"))

if num > 0:
    if num % 2 == 0:
        print("%d是偶数" % num)
    else:
        print("%d是奇数" % num)
\end{lstlisting}

\begin{tcolorbox}
	\mybox{运行结果}
	\begin{verbatim}
输入一个正整数:66
66是偶数
\end{verbatim}
\end{tcolorbox}

\newpage

\section{断言}

\subsection{断言(Assertion)}

设置断言表达式后,当满足条件时程序正常执行,当断言失败时,程序会中断执行。在进行断言时可以配置错误的提示信息,否则很难知道那块代码出现了错误。\\

通过断言可以直接查找出程序的错误,但从另外一个角度来讲,断言由于不受到程序逻辑的控制,可能会造成许多的额外的问题,在实际的开发之中慎用。\\

\mybox{断言}

\begin{lstlisting}[language=Python]
import math

print("计算三角形面积")
a = float(input("第一条边:"))
b = float(input("第二条边:"))
c = float(input("第三条边:"))

assert a + b > c, "边长不合法"
assert a + c > b, "边长不合法"
assert b + c > a, "边长不合法"

p = (a + b + c) / 2     # 半周长
area = math.sqrt(p * (p-a) * (p-b) * (p-c)) # 海伦公式
print("面积 = %.2f" % area)
\end{lstlisting}

\begin{tcolorbox}
	\mybox{运行结果}
	\begin{verbatim}
计算三角形面积
第一条边:1
第二条边:1
第三条边:2
AssertionError: 边长不合法
\end{verbatim}
\end{tcolorbox}

\newpage
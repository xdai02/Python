\chapter{分支}

\section{逻辑运算符}

\subsection{关系运算符}

编程中经常需要使用关系运算符来比较两个数据的大小,比较的结果是一个布尔值(boolean),即True(非0)或False(0)。\\

在编程中需要注意,一个等号=表示赋值运算,而两个等号==表示比较运算。\\

\begin{table}[H]
	\centering
	\setlength{\tabcolsep}{5mm}{
		\begin{tabular}{|c|c|}
			\hline
			\textbf{数学符号} & \textbf{关系运算符} \\
			\hline
			$ < $             & <                   \\
			\hline
			$ > $             & >                   \\
			\hline
			$ \le $           & <=                  \\
			\hline
			$ \ge $           & >=                  \\
			\hline
			$ = $             & ==                  \\
			\hline
			$ \ne $           & !=                  \\
			\hline
		\end{tabular}
	}
\end{table}

\vspace{0.5cm}

\subsection{逻辑运算符}

逻辑运算符用于连接多个关系表达式,其结果也是一个布尔值。\\

\begin{enumerate}
	\item 逻辑与and:当多个条件全部为True,结果为True。\\
	      \begin{table}[H]
		      \centering
		      \setlength{\tabcolsep}{5mm}{
			      \begin{tabular}{|c|c|c|}
				      \hline
				      \textbf{条件1} & \textbf{条件2} & \textbf{条件1 and 条件2} \\
				      \hline
				      T              & T              & T                        \\
				      \hline
				      T              & F              & F                        \\
				      \hline
				      F              & T              & F                        \\
				      \hline
				      F              & F              & F                        \\
				      \hline
			      \end{tabular}
		      }
	      \end{table}

	\item 逻辑或or:多个条件至少有一个为True时,结果为True。\\
	      \begin{table}[H]
		      \centering
		      \setlength{\tabcolsep}{5mm}{
			      \begin{tabular}{|c|c|c|}
				      \hline
				      \textbf{条件1} & \textbf{条件2} & \textbf{条件1 or 条件2} \\
				      \hline
				      T              & T              & T                       \\
				      \hline
				      T              & F              & T                       \\
				      \hline
				      F              & T              & T                       \\
				      \hline
				      F              & F              & F                       \\
				      \hline
			      \end{tabular}
		      }
	      \end{table}

	\item 逻辑非not:条件为True时,结果为False;条件为False时,结果为True。\\
	      \begin{table}[H]
		      \centering
		      \setlength{\tabcolsep}{5mm}{
			      \begin{tabular}{|c|c|}
				      \hline
				      \textbf{条件} & \textbf{not 条件} \\
				      \hline
				      T             & F                 \\
				      \hline
				      F             & T                 \\
				      \hline
			      \end{tabular}
		      }
	      \end{table}
\end{enumerate}

\newpage

\section{if}

\subsection{if}

if语句用于判断一个条件是否成立,如果成立则进入语句块,否则不执行。\\

\mybox{年龄}

\begin{lstlisting}[language=Python]
age = int(input("Enter your age: "))
if 0 < age < 18:
	print("Minor")
\end{lstlisting}

\begin{tcolorbox}
	\mybox{运行结果}
	\begin{verbatim}
Enter your age: 17
Minor
\end{verbatim}
\end{tcolorbox}

\vspace{0.5cm}

\subsection{if-else}

if-else的结构与if类似,只是在if语句块中的条件不成立时,执行else语句块中的语句。\\

\mybox{闰年}

\begin{lstlisting}[language=Python]
year = int(input("Enter a year: "))

"""
	A year is a leap year if it is
	1. exactly divisible by 4, and not divisible by 100;
	2. or is exactly divisible by 400
"""
if year % 4 == 0 and year % 100 != 0 or year % 400 == 0:
	print("Leap year")
else:
	print("Common year")
\end{lstlisting}

\begin{tcolorbox}
	\mybox{运行结果}
	\begin{verbatim}
Enter a year: 2020
Leap year
\end{verbatim}
\end{tcolorbox}

\vspace{0.5cm}

\subsection{if-elif-else}

当需要对更多的条件进行判断时,可以使用if-elif-else语句。\\

\mybox{字符}

\begin{lstlisting}[language=Python]
c = input("Enter a character: ")
if c >= 'a' and c <= 'z':
	print("Lowercase")
elif c >= 'A' and c <= 'Z':
	print("Uppercase")
elif c >= '0' and c <= '9':
	print("Digit")
else:
	print("Special character")
\end{lstlisting}

\begin{tcolorbox}
	\mybox{运行结果}
	\begin{verbatim}
Enter a character: T
Uppercase
\end{verbatim}
\end{tcolorbox}

\newpage
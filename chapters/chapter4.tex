\chapter{序列}

\section{列表}

\subsection{列表(List)}

列表用于存储多个数据,使用[]或list()函数创建。列表中的元素可以通过下标来访问,下标从0开始。列表除了正向索引访问之外,也可以进行反向索引访问。

\vspace{-0.5cm}

\begin{lstlisting}[language=Python]
lst = [1, 2, 3]

print(lst[0])		# 1
print(lst[1])		# 2
print(lst[2])		# 3

print(lst[-1])		# 3
print(lst[-2])		# 2
print(lst[-3])		# 1

print(lst[3])       # IndexError
\end{lstlisting}

+运算符可以用于列表的拼接,*运算符可以用于列表的重复。

\vspace{-0.5cm}

\begin{lstlisting}[language=Python]
lst = [1, 2, 3] + [4, 5, 6]
print(lst)      # [1, 2, 3, 4, 5, 6]

lst = [1, 2, 3] * 3
print(lst)      # [1, 2, 3, 1, 2, 3, 1, 2, 3]
\end{lstlisting}

\vspace{0.5cm}

\subsection{in}

in运算符用于判断某个元素是否在序列中,如果在则返回True,否则返回False。\\

\mybox{查询}

\begin{lstlisting}[language=Python]
languages = ["C", "C++", "Python", "Java"]
key = input("Enter a language: ")

if key in languages:
	print("Found")
else:
	print("Not found")
\end{lstlisting}

\begin{tcolorbox}
	\mybox{运行结果}
	\begin{verbatim}
Enter a language: Python
Found
\end{verbatim}
\end{tcolorbox}

\vspace{0.5cm}

\subsection{切片(Slicing)}

切片用于截取列表中的一部分元素,切片使用[start:end:step]进行操作,其中:

\begin{itemize}
	\item start:切片开始下标(包含),默认为0
	\item end:切片结束下标(不包含),默认为列表长度
	\item step:切片的步长,默认为1
\end{itemize}

\vspace{-0.5cm}

\begin{lstlisting}[language=Python]
lst = list(range(10))

print(lst)			# [0, 1, 2, 3, 4, 5, 6, 7, 8, 9]
print(lst[2:7])		# [2, 3, 4, 5, 6]
print(lst[:5])		# [0, 1, 2, 3, 4]
print(lst[3:])		# [3, 4, 5, 6, 7, 8, 9]
print(lst[:])		# [0, 1, 2, 3, 4, 5, 6, 7, 8, 9]
print(lst[::2])		# [0, 2, 4, 6, 8]
\end{lstlisting}

\vspace{0.5cm}

\subsection{列表方法}

列表提供了很多内置方法,可以用于处理列表中的数据。\\

\begin{table}[H]
	\centering
	\setlength{\tabcolsep}{5mm}{
		\begin{tabular}{|l|l|}
			\hline
			\textbf{方法} & \textbf{功能}                  \\
			\hline
			append()      & 追加数据                       \\
			\hline
			extend()      & 追加列表                       \\
			\hline
			insert()      & 指定位置插入                   \\
			\hline
			remove()      & 删除第一个出现的指定数据       \\
			\hline
			pop()         & 删除指定位置的数据             \\
			\hline
			index()       & 查询指定数据第一次出现的位置   \\
			\hline
			count()       & 统计指定数据在列表中出现的次数 \\
			\hline
			sort()        & 对列表进行排序                 \\
			\hline
			reverse()     & 对列表进行反转                 \\
			\hline
			clear()       & 清空列表                       \\
			\hline
		\end{tabular}
	}
	\caption{列表操作函数}
\end{table}

\vspace{0.5cm}

\mybox{列表方法}

\begin{lstlisting}[language=Python]
lst = list(range(5))
print("lst =", lst)

lst.append(5)
print("append(5):", lst)

lst.insert(0, 8)
print("insert(0, 8):", lst)

lst.extend([8, 2, 3])
print("extend([8, 2, 3]):", lst)

lst.remove(5)
print("remove(5):", lst)

lst.pop(0)
print("pop(0):", lst)

print("index(3):", lst.index(3))

print("count(8):", lst.count(8))

lst.sort()
print("sort():", lst)

lst.sort(reverse=True)
print("sort(reverse=True):", lst)

lst.reverse()
print("reverse():", lst)

lst.clear()
print("clear():", lst)
\end{lstlisting}

\begin{tcolorbox}
	\mybox{运行结果}
	\begin{verbatim}
lst = [0, 1, 2, 3, 4]
append(5): [0, 1, 2, 3, 4, 5]
insert(0, 8): [8, 0, 1, 2, 3, 4, 5]
extend([8, 2, 3]): [8, 0, 1, 2, 3, 4, 5, 8, 2, 3]
remove(5): [8, 0, 1, 2, 3, 4, 8, 2, 3]
pop(0): [0, 1, 2, 3, 4, 8, 2, 3]
index(3): 3
count(8): 1
sort(): [0, 1, 2, 2, 3, 3, 4, 8]
sort(reverse=True): [8, 4, 3, 3, 2, 2, 1, 0]
reverse(): [8, 4, 3, 3, 2, 2, 1, 0]
clear(): []
\end{verbatim}
\end{tcolorbox}

\newpage

\section{元组}

\subsection{元组(Tuple)}

元组与列表类似,但是元组中的元素是不可修改的。元素使用()或tuple()定义,当元组只有一个元素时,必须在元素后面加上逗号。\\

\mybox{两点间距离}

\begin{lstlisting}[language=Python]
import math

p1 = (0, 0)
p2 = (3, 4)
distance = math.sqrt((p1[0] - p2[0]) ** 2 + (p1[1] - p2[1]) ** 2)
print(distance)
\end{lstlisting}

\begin{tcolorbox}
	\mybox{运行结果}
	\begin{verbatim}
5.0
\end{verbatim}
\end{tcolorbox}

\vspace{0.5cm}

\subsection{序列统计函数}

\begin{table}[H]
	\centering
	\setlength{\tabcolsep}{5mm}{
		\begin{tabular}{|l|l|}
			\hline
			\textbf{函数} & \textbf{功能}                              \\
			\hline
			len()         & 获取序列的长度                             \\
			\hline
			max()         & 获取序列中的最大值                         \\
			\hline
			min()         & 获取序列中的最小值                         \\
			\hline
			sum()         & 计算序列中的内容总和                       \\
			\hline
			any()         & 序列中有一个为True结果为True,否则为False  \\
			\hline
			all()         & 序列中有一个为False结果为False,否则为True \\
			\hline
		\end{tabular}
	}
	\caption{序列统计函数}
\end{table}

\begin{lstlisting}[language=Python]
lst = [4, 0, 1, 3, 2]
tup = (8, 5, 7, 9)

print(len(lst))		# 5
print(len(tup))		# 4

print(max(lst))		# 4
print(max(tup))		# 9

print(min(lst))		# 0
print(min(tup))		# 5

print(sum(lst))		# 10
print(sum(tup))		# 29
\end{lstlisting}

\newpage

\section{集合}

\subsection{集合(Set)}

集合表示一组无序且不重复的元素,使用{}或set()定义。\\

集合是无序的,因此不能通过下标来访问集合中的元素,但是可以通过in来判断元素是否在集合中。\\

集合支持数学上的集合运算,包括交集、并集、差集等:

\begin{itemize}
	\item 交集:intersection()或\&
	\item 并集:union()或|
	\item 差集:difference()或-
\end{itemize}

\vspace{-0.5cm}

\begin{lstlisting}[language=Python]
s1 = {1, 2, 3}
s2 = {3, 4, 5}

print(s1 & s2)				# {3}
print(s1.intersection(s2))	# {3}

print(s1 | s2)				# {1, 2, 3, 4, 5}
print(s1.union(s2))			# {1, 2, 3, 4, 5}

print(s1 - s2)				# {1, 2}
print(s1.difference(s2))	# {1, 2}
\end{lstlisting}

\newpage

\section{字符串}

\subsection{字符串修改}

\begin{table}[H]
	\centering
	\setlength{\tabcolsep}{5mm}{
		\begin{tabular}{|l|l|}
			\hline
			\textbf{方法} & \textbf{功能}    \\
			\hline
			lower()       & 转换小写         \\
			\hline
			upper()       & 转换大写         \\
			\hline
			capitalize()  & 首字母大写       \\
			\hline
			strip()       & 去除首尾空白字符 \\
			\hline
			replace()     & 字符串替换       \\
			\hline
		\end{tabular}
	}
\end{table}

\mybox{字符串修改}

\begin{lstlisting}[language=Python]
s = "Hello World!"

print("[Lower]")
print(s.lower())

print("[Upper]")
print(s.upper())

print("[Capitalize]")
print(s.capitalize())

print("[Strip]")
print("   Hello World!\n \t".strip())

print("[Replace]")
print(s.replace("Hello", "Bye"))
\end{lstlisting}

\begin{tcolorbox}
	\mybox{运行结果}
	\begin{verbatim}
[Lower]
hello world!
[Upper]     
HELLO WORLD!
[Capitalize]
Hello world!
[Strip]     
Hello World!
[Replace]
Bye World!
\end{verbatim}
\end{tcolorbox}

\vspace{0.5cm}

\subsection{字符串分割}

\begin{table}[H]
	\centering
	\setlength{\tabcolsep}{5mm}{
		\begin{tabular}{|l|l|}
			\hline
			\textbf{方法} & \textbf{功能} \\
			\hline
			join()        & 字符串拼接    \\
			\hline
			split()       & 字符串分割    \\
			\hline
		\end{tabular}
	}
\end{table}

\mybox{日期}

\begin{lstlisting}[language=Python]
date_time = "2023/1/14 23:26:51"

date, time = date_time.split(" ")

year, month, day = date.split("/")
hour, minute, second = time.split(":")

date = [day, month, year]
date = "/".join(date)

if int(hour) < 12:
	time = [hour, minute, second]
	time = ":".join(time) + " AM"
else:
	time = [str(int(hour) - 12), minute, second]
	time = ":".join(time) + " PM"

date_time = date + " " + time
print(date_time)
\end{lstlisting}

\begin{tcolorbox}
	\mybox{运行结果}
	\begin{verbatim}
14/1/2023 11:26:51 PM
\end{verbatim}
\end{tcolorbox}

\newpage

\section{字典}

\subsection{字典(Dictionary)}

字典是一个在开发之中极为重要的类型,字典提供了非常方便地数据内容查找操作。字典的本质就是一个数据查找的序列,与之前的列表或者元组不同的地方在于,其它的结构保存数据的目的都是为了输出使用,而字典是为了查询使用。\\

字典严格意义上来讲属于一种哈希表的结构,在哈希表进行数据存储的时候往往都需要一个哈希算法进行数据存储位置的计算。\\

字典是一个二元偶对象的集合,所以里面保存的数据都是成对的,所有的数据内容都按照key = value的形式进行存放。考虑到用户使用的方便,key的类型可以使数字、字符串或者是元组,但是最为常见的还是字符串。\\

在使用字典进行查询的时候如果指定的key不存在,则在代码执行会出现KeyError错误提示信息,也就是原生的数据查询方式是有可能会抛出异常的。\\

字典的本质是根据key查找对应的value,一旦key重复的时候,使用新的数据覆盖掉旧的数据。\\

\mybox{字典}

\begin{lstlisting}[language=Python]
info = {"name": "小灰", "age": 16, "height": 175.6}
print(info)
print(info["age"])

info["height"] = 180
print(info)
\end{lstlisting}

\begin{tcolorbox}
	\mybox{运行结果}
	\begin{verbatim}
{'name': '小灰', 'age': 16, 'height': 175.6}
16
{'name': '小灰', 'age': 16, 'height': 180}
\end{verbatim}
\end{tcolorbox}

既然字典是一个序列,那么字典可以使用in进行判断指定key是否存在,通过这样的机制可以避免由于key不存在所带来的KeyError异常的抛出。\\


同样,字典也可以使用for循环迭代输出,但是迭代的知识字典中全部的key。\\

\mybox{字典迭代输出}

\begin{lstlisting}[language=Python]
info = {"name": "小灰", "age": 16, "height": 175.6}
for key in info:
    print("%s: %s" % (key, info[key]))
\end{lstlisting}

\begin{tcolorbox}
	\mybox{运行结果}
	\begin{verbatim}
name: 小灰
age: 16
height: 175.6
\end{verbatim}
\end{tcolorbox}

在进行字典数据迭代输出的时候,最好的做法是每一次迭代直接返回当前完整的key:value的映射项,此时的处理就需要依赖items()来完成。\\

\mybox{items()迭代输出}

\begin{lstlisting}[language=Python]
info = {"name": "小灰", "age": 16, "height": 175.6}
for key, value in info.items():
    print("%s: %s" % (key, value))
\end{lstlisting}

\begin{tcolorbox}
	\mybox{运行结果}
	\begin{verbatim}
name: 小灰
age: 16
height: 175.6
\end{verbatim}
\end{tcolorbox}

\vspace{0.5cm}

\subsection{字典操作函数}

\begin{table}[H]
	\centering
	\setlength{\tabcolsep}{5mm}{
		\begin{tabular}{|l|l|}
			\hline
			\textbf{函数}            & \textbf{功能}             \\
			\hline
			clear()                  & 清空字典数据              \\
			\hline
			update(\{k:v, ...\})     & 更新字典数据              \\
			\hline
			get(key[, defaultvalue]) & 根据key获取数据           \\
			\hline
			pop(key)                 & 弹出字典中指定的key数据   \\
			\hline
			popitem()                & 从字典中弹出一组映射项    \\
			\hline
			keys()                   & 返回字典中全部key数据     \\
			\hline
			values()                 & 返回字典中全部的value数据 \\
			\hline
		\end{tabular}
	}
	\caption{字典操作函数}
\end{table}

\subsubsection{字典数据更新update()}

Python中的字典是可以进行存储数据的动态扩充的,update()除了拥有表面上的更新之外,也拥有数据的扩充操作。\\

\mybox{字典数据更新update()}

\begin{lstlisting}[language=Python]
info = {"name": "小灰", "age": 16, "height": 175.6}
info.update({"city": "Shanghai", "job": "programmer"})
print(info)
\end{lstlisting}

\begin{tcolorbox}
	\mybox{运行结果}
	\begin{verbatim}
{'name': '小灰', 'age': 16, 'height': 175.6,
'city': 'Shanghai', 'job': 'programmer'}
\end{verbatim}
\end{tcolorbox}

\subsubsection{字典数据删除pop()}

pop()可以根据key进行弹出操作。\\

\mybox{字典数据删除}

\begin{lstlisting}[language=Python]
info = {"name": "小灰", "age": 16, "height": 175.6}
print(info.pop("age"))
print(info)
\end{lstlisting}

\begin{tcolorbox}
	\mybox{运行结果}
	\begin{verbatim}
16
{'name': '小灰', 'height': 175.6}
\end{verbatim}
\end{tcolorbox}

\subsubsection{字典数据获取get()}

可以通过key获取数据,如果key不存在则返回None。\\

\mybox{字典数据获取}

\begin{lstlisting}[language=Python]
info = {"name": "小灰", "age": 16, "height": 175.6}
print(info.get("name"))
print(info.get("job"))
\end{lstlisting}

\begin{tcolorbox}
	\mybox{运行结果}
	\begin{verbatim}
小灰
None
\end{verbatim}
\end{tcolorbox}

\newpage
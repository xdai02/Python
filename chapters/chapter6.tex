\chapter{模块}

\section{random}

\subsection{random}

random模块提供了生成随机数据的功能。

\begin{table}[H]
    \centering
    \setlength{\tabcolsep}{5mm}{
        \begin{tabular}{|l|l|}
            \hline
            \textbf{方法} & \textbf{功能}                      \\
            \hline
            random()      & 生成一个$ [0, 1] $之间的随机浮点数 \\
            \hline
            randint(x, y) & 生成一个$ [x, y] $之间的随机整数   \\
            \hline
            choice()      & 从序列中随机返回一个元素           \\
            \hline
            shuffle()     & 将序列打乱                         \\
            \hline
            sample()      & 从序列中生成一组唯一的随机元素     \\
            \hline
        \end{tabular}
    }
    \caption{random模块}
\end{table}

\mybox{随机密码生成}

\begin{lstlisting}[language=Python]
import random
import string

def password_generator(length):
    characters = string.ascii_letters + string.digits
    return "".join(random.choice(characters) for _ in range(length))

def main():
    length = int(input("Enter length of password: "))
    print(password_generator(length))

if __name__ == "__main__":
    main()
\end{lstlisting}

\begin{tcolorbox}
    \mybox{运行结果}
    \begin{verbatim}
Enter length of password: 8
@o8VBuiV
\end{verbatim}
\end{tcolorbox}

\newpage

\section{copy}

\subsection{copy}

copy是一个专门进行内容复制的处理模块。拷贝分为浅拷贝(shallow copy)和深拷贝(deep copy)两种:

\begin{itemize}
    \item 浅拷贝:只是复制第一层的内容,而更深入的数据嵌套关系不会拷贝。
    \item 深拷贝:会进行完整的复制。
\end{itemize}

\vspace{0.5cm}

\subsection{引用}

引用的本质在于将同一块内存空间,交给不同的对象进行同时操作,当一个对象修改了内存数据之后,其它对象的内存数据也会同时发生改变。

\vspace{-0.5cm}

\begin{lstlisting}[language=Python]
a = {1:[1, 2, 3]}
b = a
\end{lstlisting}

\begin{figure}[H]
    \centering
    \begin{tikzpicture}
        \draw (0,0) rectangle node{\{1:\textcolor{blue}{[1, 2, 3]}\}} (2.5,1);
        \draw (-2,1.5) circle(0.5) node{a};
        \draw (-2,-0.5) circle(0.5) node{b};
        \draw (3,-2) rectangle node{\textcolor{blue}{[1, 2, 3]}} (5,-1);

        \draw[->] (-1.5,1.5) -- (0,1);
        \draw[->] (-1.5,-0.5) -- (0,0);
        \draw[->, blue] (1.4,0.2) -- (1.4,-1.5) -- (3,-1.5);
    \end{tikzpicture}
    \caption{引用}
\end{figure}

\vspace{0.5cm}

\mybox{引用传递}

\begin{lstlisting}[language=Python]
def main():
    info = dict(name="小灰", age=16, skills=["Python", "C/C++"])
    copy_info = info        # 引用传递
    copy_info["skills"].append("Java")
    print(info)

if __name__ == "__main__":
    main()
\end{lstlisting}

\begin{tcolorbox}
    \mybox{运行结果}
    \begin{verbatim}
{'name':'小灰', 'age':16, 'skills':['Python', 'C/C++', 'Java']}
\end{verbatim}
\end{tcolorbox}

\vspace{0.5cm}

\subsection{浅拷贝}

拷贝和引用传递是不同的,拷贝是将原始的内存的数据进行一份复制,而后为其分配单独的对象的指向。

\vspace{-0.5cm}

\begin{lstlisting}[language=Python]
a = {1:[1, 2, 3]}
b = a.copy()
\end{lstlisting}

\begin{figure}[H]
    \centering
    \begin{tikzpicture}
        \draw (0,1) rectangle node{\{1:\textcolor{blue}{[1, 2, 3]}\}} (2.5,2);
        \draw (-2,1.5) circle(0.5) node{a};
        \draw (0,-1) rectangle node{\{1:\textcolor{red}{[1, 2, 3]}\}} (2.5,0);
        \draw (-2,-0.5) circle(0.5) node{b};
        \draw (4,0) rectangle node{[1, 2, 3]} (6,1);

        \draw[->] (-1.5,1.5) -- (0,1.5);
        \draw[->] (-1.5,-0.5) -- (0,-0.5);
        \draw[->, blue] (1.4,1.2) -- (1.4,0.75) -- (4,0.75);
        \draw[->, red] (1.4,-0.2) -- (1.4,0.3) -- (4,0.3);
    \end{tikzpicture}
    \caption{浅拷贝}
\end{figure}

\vspace{0.5cm}

\mybox{浅拷贝}

\begin{lstlisting}[language=Python]
import copy

def main():
    info = dict(name="小灰", age=16, skills=["Python", "C/C++"])
    copy_info = copy.copy(info)     # 浅拷贝
    copy_info.pop("age")
    copy_info["skills"].append("Java")
    print(info)
    print(copy_info)

if __name__ == "__main__":
    main()
\end{lstlisting}

\begin{tcolorbox}
    \mybox{运行结果}
    \begin{verbatim}
{'name':'小灰', 'age':16, 'skills':['Python', 'C/C++', 'Java']}
{'name':'小灰', 'skills':['Python', 'C/C++', 'Java']}
\end{verbatim}
\end{tcolorbox}

\vspace{0.5cm}

\subsection{深拷贝}

\vspace{-0.5cm}

\begin{lstlisting}[language=Python]
a = {1:[1, 2, 3]}
b = copy.deepcopy(a)
\end{lstlisting}

\begin{figure}[H]
    \centering
    \begin{tikzpicture}
        \draw (0,1) rectangle node{\{1:\textcolor{blue}{[1, 2, 3]}\}} (2.5,2);
        \draw (-2,1.5) circle(0.5) node{a};
        \draw (0,-1) rectangle node{\{1:\textcolor{red}{[1, 2, 3]}\}} (2.5,0);
        \draw (-2,-0.5) circle(0.5) node{b};
        \draw (4,0) rectangle node{\textcolor{blue}{[1, 2, 3]}} (6,1);
        \draw (4,-2) rectangle node{\textcolor{red}{[1, 2, 3]}} (6,-1);

        \draw[->] (-1.5,1.5) -- (0,1.5);
        \draw[->] (-1.5,-0.5) -- (0,-0.5);
        \draw[->, blue] (1.4,1.2) -- (1.4,0.5) -- (4,0.5);
        \draw[->, red] (1.4,-0.8) -- (1.4,-1.5) -- (4,-1.5);
    \end{tikzpicture}
    \caption{深拷贝}
\end{figure}

\vspace{0.5cm}

\mybox{深拷贝}

\begin{lstlisting}[language=Python]
import copy

def main():
    info = dict(name="小灰", age=16, skills=["Python", "C/C++"])
    copy_info = copy.deepcopy(info)     # 深拷贝
    copy_info.pop("age")
    copy_info["skills"].append("Java")
    print(info)
    print(copy_info)

if __name__ == "__main__":
    main()
\end{lstlisting}

\begin{tcolorbox}
    \mybox{运行结果}
    \begin{verbatim}
{'name': '小灰', 'age': 16, 'skills': ['Python', 'C/C++']}
{'name': '小灰', 'skills': ['Python', 'C/C++', 'Java']}
\end{verbatim}
\end{tcolorbox}

\newpage

\section{MapReduce数据处理}

\subsection{MapReduce}

Python在数据分析领域上使用非常广泛,并且实现简单。在Python中可以进行大量数据的快速处理,进行数据的过滤、分析操作。在数据量小的情况下可以方便地使用for循环进行数据的逐个处理,但是在数据量大的情况下,就需要使用一些特定的处理函数进行过滤、分析、统计等操作。\\

在Python中默认提供有了filter()、map(),但是要进行统计处理,则需要导入reduce()。

\begin{table}[H]
    \centering
    \setlength{\tabcolsep}{5mm}{
        \begin{tabular}{|l|l|}
            \hline
            \textbf{函数}              & \textbf{功能}            \\
            \hline
            filter(function, sequence) & 对传入的序列数据进行过滤 \\
            \hline
            map(function, sequence)    & 对传入的序列数据进行处理 \\
            \hline
            reduce(function, sequence) & 对传入的序列数据进行统计 \\
            \hline
        \end{tabular}
    }
    \caption{MapReduce数据处理函数}
\end{table}

在进行数据处理的过程之中都需要有一个处理函数,这个处理函数就定义了数据该如何进行处理或统计,一般而言这样的函数都比较短,所以大部分情况下都可以利用lambda函数来完成。\\

\mybox{MapReduce数据处理}

\begin{lstlisting}[language=Python]
from functools import reduce

def main():
    lst = list(range(10))

    filter_lst = list(filter(lambda x: x % 2 ==0, lst))
    print("过滤出偶数:%s" % filter_lst)

    map_lst = list(map(lambda x: x ** 2, filter_lst))
    print("平方:%s" % map_lst)

    result = reduce(lambda x, y: x+y, map_lst)
    print("求和:%d" % result)

if __name__ == "__main__":
    main()
\end{lstlisting}

\begin{tcolorbox}
    \mybox{运行结果}
    \begin{verbatim}
过滤出偶数:[0, 2, 4, 6, 8]
平方:[0, 4, 16, 36, 64]
求和:120
\end{verbatim}
\end{tcolorbox}

\newpage

\section{jieba}

\subsection{pip}

Python本地有一些系统模块开发者可以直接进行使用,但是开发者仅仅是依靠系统模块是不够的,需要大量的使用第三方模块。为了解决这些模块的管理问题,在Python中内置了pip管理工具。通过此工具可以直接连接到Python远程服务模块仓库,通过仓库下载所需要的模块。\\

在Python安装的时候会自动进行pip工具的相关安装,输入pip --help查看pip的相关命令选项。

\begin{table}[H]
    \centering
    \setlength{\tabcolsep}{5mm}{
        \begin{tabular}{|l|l|}
            \hline
            \textbf{功能}  & \textbf{命令}                \\
            \hline
            搜索模块       & pip search 模块名            \\
            \hline
            安装模块       & pip install 模块名           \\
            \hline
            查看已安装模块 & pip list                     \\
            \hline
            列出过期模块   & pip list --outdated          \\
            \hline
            更新模块       & pip install --upgrade 模块名 \\
            \hline
            卸载模块       & pip uninstall 模块名         \\
            \hline
        \end{tabular}
    }
    \caption{pip命令}
\end{table}

\vspace{0.5cm}

\subsection{jieba}

分词是一种数学的应用,它可以直接根据词语之间的数学关系进行文字或单词的抽象。例如对“中华人民共和国”进行分词处理,可以拆分为“中华”、“华人”、“人民”、“共和”、“共和国”、“中华人民共和国”。如果没有分词,就无法进行搜索引擎的开发。\\

jieba是在中文自然语言处理中用得最多的工具包之一,它以分词起家,目前已经能够实现包括分词、词性标注以及命名实体识别等多种功能。\\

考虑到分词的效果与性能,在jieba组件中提供有3种分词模式:

\begin{enumerate}
    \item 精确模式:将句子进行最精确的切分,分词速度相对较低。

    \item 全模式:基于词汇列表将句子中所有可以成词的词语都扫描出来,该模式处理速度非常快,但是不能有效解决歧义的问题。

    \item 搜索引擎模式:在精确模式的基础上,对长词进行再次切分,该模式适用于搜索引擎构建索引的分词。
\end{enumerate}

\vspace{0.5cm}

\mybox{统计《西游记》中出现次数最多的20个词语}

\begin{lstlisting}[language=Python]
import jieba

PATH = "西游记.txt"     # 文件路径

def main():
    word_frequence = {}     # 词频表

    # 打开文件
    with open(file=PATH, mode="r", encoding="UTF-8") as file:
        line = file.readline()  # 读取一行数据
        while line:
            words = jieba.lcut(line)    # 分词
            for word in words:
                if len(word) == 1:  # 舍弃长度为1的词
                    continue
                else:
                    # dict.get(key, default=None)
                    word_frequence[word] 
                        = word_frequence.get(word, 0) + 1
            line = file.readline()

        # 获取所有数据项
        items = list(word_frequence.items())
        # 根据出现次数降序排序
        items.sort(key=lambda x: x[1], reverse=True)

        # 取前20项
        for i in range(20):
            word, count = items[i]
            print("%s: %s" % (word, count))

if __name__ == "__main__":
    main()
\end{lstlisting}

\newpage
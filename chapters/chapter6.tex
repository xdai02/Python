\chapter{模块}

\section{模块导入}

\subsection{模块(Module)}

模块是进行大型项目拆分组织的一种有效技术手段,它可以将一个庞大的代码分割成若干个小的组成单元,方便进行代码的开发与维护。利用模块的划分,在进行代码维护的时候,可以保证局部的更新不影响其它的程序的运行操作。\\

使用import关键字可以进行模块的导入,import可以同时导入多个模块,但是从开发的角度来讲,强烈建议分开导入。

\vspace{-0.5cm}

\begin{lstlisting}[language=Python]
import package.module [as alias] [, ...]
\end{lstlisting}

\vspace{0.5cm}

\mybox{模块导入}

\begin{lstlisting}[language=Python, title=algorithm/search.py]
def sequence_search(list, key):
    """
        顺序查找
        Args:
            list (list): 待查找数组
            key (int): 关键字
    """
    for i in range(len(list)):
        if list[i] == key:
            return i
    return -1

def binary_search(list, key):
    """
        二分查找
        Args:
            list (list): 待查找数组
            key (int): 关键字
    """
    start = 0
    end = len(list) - 1
    while start <= end:
        mid = (start + end) // 2
        if list[mid] == key:
            return mid
        elif list[mid] < key:
            start = mid + 1
        else:
            end = mid - 1
    return -1
\end{lstlisting}

\begin{lstlisting}[language=Python, title=import\_as.py]
import algorithm.search as search

def main():
    list = [40, 9, 20, 93, 7, 34, 85, 91]
    key = 34
    print("%d所在位置:%d" % (key, search.sequence_search(list, key)))

if __name__ == "__main__":
    main()
\end{lstlisting}

\begin{tcolorbox}
	\mybox{运行结果}
	\begin{verbatim}
34所在位置:5
\end{verbatim}
\end{tcolorbox}

在Python里有一个作者写的Python开发禅道(开发的19条哲学),如果想看到这个彩蛋的信息,可以在Python的交互模式输入import this。\\

\vspace{-0.5cm}

\begin{tcolorbox}
	\mybox{运行结果}
	\begin{verbatim}
The Zen of Python, by Tim Peters

Beautiful is better than ugly.
Explicit is better than implicit.
Simple is better than complex.
Complex is better than complicated.
Flat is better than nested.
Sparse is better than dense.
Readability counts.
Special cases aren't special enough to break the rules.
Although practicality beats purity.
Errors should never pass silently.
Unless explicitly silenced.
In the face of ambiguity, refuse the temptation to guess.
There should be one-- and preferably only one --obvious way to
do it.
Although that way may not be obvious at first unless you're
Dutch.
Now is better than never.
Although never is often better than *right* now.
If the implementation is hard to explain, it's a bad idea.
If the implementation is easy to explain, it may be a good idea.
Namespaces are one honking great idea -- let's do more of those!
\end{verbatim}
\end{tcolorbox}

\vspace{0.5cm}

\subsection{from-import模块导入}

使用import导入模块之后需要采用module\_name.func()形式进行调用,每一次调用函数非常麻烦。from-import导入语法可以简化调用语句。\\

\vspace{-0.5cm}

\begin{lstlisting}[language=Python]
from package.module import name [as alias] [, ...]
\end{lstlisting}

在实践中, from-import不是良好的编程风格,因为如果导入的变量与作用域中现有变量同名,那么变量就会被悄悄覆盖掉。使用import语句的时候就不会发生这种问题,通过module.var或module.func()获取的变量或方法不会与现有作用域冲突。\\

\mybox{from-import导入}

\begin{lstlisting}[language=Python, title=algorithm/search.py]
def sequence_search(list, key):
    """
        顺序查找
        Args:
            list (list): 待查找数组
            key (int): 关键字
    """
    for i in range(len(list)):
        if list[i] == key:
            return i
    return -1

def binary_search(list, key):
    """
        二分查找
        Args:
            list (list): 待查找数组
            key (int): 关键字
    """
    start = 0
    end = len(list) - 1
    while start <= end:
        mid = (start + end) // 2
        if list[mid] == key:
            return mid
        elif list[mid] < key:
            start = mid + 1
        else:
            end = mid - 1
    return -1
\end{lstlisting}

\begin{lstlisting}[language=Python, title=from\_import.py]
from algorithm.search import binary_search

def main():
    list = [7, 9, 20, 34, 40, 85, 91, 93]
    key = 34
    print("%d所在位置:%d" % (key, binary_search(list, key)))

if __name__ == "__main__":
    main()
\end{lstlisting}

\begin{tcolorbox}
	\mybox{运行结果}
	\begin{verbatim}
34所在位置:3
\end{verbatim}
\end{tcolorbox}

\newpage

\section{math模块}

\subsection{math模块}

现代计算机的基础学科就是数学,如果没有数学理论作为基础,计算机是无法的得到正常发展的。数学模块只提供了数学的基本计算功能,在很多的开发之中,有可能会需要使用到更加复杂的数学逻辑的时候就需要采用一些第三方模块进行数学计算了。\\

\mybox{math模块}

\begin{lstlisting}[language=Python]
import math

def main():
    print("累加:%d" % math.fsum(range(101)))
    print("阶乘:%d" % math.factorial(10))
    print("乘方:%d" % math.pow(2, 10))
    print("对数:%f" % math.log(10))
    print("余数:%d" % math.fmod(22, 5))

if __name__ == "__main__":
    main()
\end{lstlisting}

\begin{tcolorbox}
	\mybox{运行结果}
	\begin{verbatim}
累加:5050
阶乘:3628800
乘方:1024
对数:2.302585
余数:2
\end{verbatim}
\end{tcolorbox}

\newpage

\section{random模块}

\subsection{random模块}

随机数可以在一个指定的范围之内随机地生成一些数字供使用。例如,手机验证码发送来的数字就是使用随机数的方式生成的。

\begin{table}[H]
	\centering
	\setlength{\tabcolsep}{5mm}{
		\begin{tabular}{|l|l|}
			\hline
			\textbf{方法}         & \textbf{功能}                                   \\
			\hline
			random()              & 生成一个0到1的随机浮点数:$ 0.0 \le n \le 1.0 $ \\
			\hline
			uniform(x, y)         & 生成一个在指定范围内的随机浮点数                \\
			\hline
			randint(x, y)         & 生成一个指定范围内的随机整数$ x \le n \le y $   \\
			\hline
			choice(sequence)      & 从序列中随机抽取数据                            \\
			\hline
			shuffle(x [, random]) & 将一个列表中的元素打乱                          \\
			\hline
			sample(sequence, k)   & 从指定序列中随机获取指定序列分片                \\
			\hline
		\end{tabular}
	}
	\caption{random模块}
\end{table}

\mybox{random模块}

\begin{lstlisting}[language=Python]
import random

def main():
    lst = [random.randint(1, 100) for _ in range(10)]
    print("初始序列:%s" % lst)
    print("随机抽取:", end='')
    for _ in range(5):
        print(random.choice(lst), end=' ')

if __name__ == "__main__":
    main()
\end{lstlisting}

\begin{tcolorbox}
	\mybox{运行结果}
	\begin{verbatim}
初始序列:[85, 83, 83, 29, 2, 93, 30, 65, 41, 54]
随机抽取:41 93 2 54 93
\end{verbatim}
\end{tcolorbox}

\newpage

\section{time模块}

\subsection{time模块}

time模块是Python内置的一个实现时间的操作模块,用于描述日期时间的数据类型分为三种:

\subsubsection{时间戳(timestamp)}

从1970年1月1日00时00分00秒开始的按秒计算的时间偏移量。\\

\mybox{计算操作耗时}

\begin{lstlisting}[language=Python]
import time

def main():
    start = time.time()
    print("【开始】%s" % start)
    sum = 0
    for i in range(99999999):
        sum += i
    end = time.time()
    print("【结束】%s" % end)
    print("【耗时】%.2fs" % (end - start))

if __name__ == "__main__":
    main()
\end{lstlisting}

\begin{tcolorbox}
	\mybox{运行结果}
	\begin{verbatim}
【开始】1617327695.1819823
【结束】1617327701.1566718
【耗时】5.97s
\end{verbatim}
\end{tcolorbox}

\subsubsection{时间元组}

保存日期时间数据的元素结构对象。

\begin{table}[H]
	\centering
	\setlength{\tabcolsep}{5mm}{
		\begin{tabular}{|c|l|l|}
			\hline
			\textbf{属性} & \textbf{功能} & \textbf{数值}                 \\
			\hline
			tm\_year      & 年            & yyyy                          \\
			\hline
			tm\_mon       & 月            & 1 $ \sim $ 12                 \\
			\hline
			tm\_mday      & 日            & 1 $ \sim $ 32                 \\
			\hline
			tm\_hour      & 时            & 0 $ \sim $ 23                 \\
			\hline
			tm\_min       & 分            & 0 $ \sim $ 59                 \\
			\hline
			tm\_sec       & 秒            & 0 $ \sim $ 61(60或61是闰秒) \\
			\hline
			tm\_wday      & 一周第几天    & 0 $ \sim $ 6(0表示周一)     \\
			\hline
			tm\_yday      & 一年第几天    & 1 $ \sim $ 366                \\
			\hline
		\end{tabular}
	}
	\caption{时间元组}
\end{table}

\vspace{0.5cm}

\mybox{时间戳与时间元组的转换}

\begin{lstlisting}[language=Python]
import time

def main():
    current_time = time.time()
    current_time_tuple = time.localtime(current_time)
    print("时间戳转换时间元组:" + str(current_time_tuple))

if __name__ == "__main__":
    main()
\end{lstlisting}

\begin{tcolorbox}
	\mybox{运行结果}
	\begin{verbatim}
时间戳转换时间元组:time.struct_time(tm_year=2021, tm_mon=4,
tm_mday=2, tm_hour=9, tm_min=45, tm_sec=16, tm_wday=4,
tm_yday=92, tm_isdst=0)
\end{verbatim}
\end{tcolorbox}

\subsubsection{格式化日期时间}

可以按照指定的标记进行格式化处理。时间戳与时间元组更多情况下还是描述程序层次上的概念,格式化日期时间可以给出人们都认可的显示格式。\\

\mybox{格式化日期时间}

\begin{lstlisting}[language=Python]
import time

def main():
    current_time = time.time()
    current_time_tuple = time.localtime(current_time)

    print(time.strftime("%Y-%m-%d %H:%M:%S", current_time_tuple))
    print("date: %s" % time.strftime("%F", current_time_tuple))
    print("time: %s" % time.strftime("%T", current_time_tuple))

if __name__ == "__main__":
    main()
\end{lstlisting}

\begin{tcolorbox}
	\mybox{运行结果}
	\begin{verbatim}
2021-04-02 09:49:01
date: 2021-04-02
time: 09:49:01
\end{verbatim}
\end{tcolorbox}

\newpage

\section{copy模块}

\subsection{copy模块}

copy是一个专门进行内容复制的处理模块。拷贝分为浅拷贝(shallow copy)和深拷贝(deep copy)两种:

\begin{itemize}
	\item 浅拷贝:只是复制第一层的内容,而更深入的数据嵌套关系不会拷贝。
	\item 深拷贝:会进行完整的复制。
\end{itemize}

\vspace{0.5cm}

\subsection{引用}

引用的本质在于将同一块内存空间,交给不同的对象进行同时操作,当一个对象修改了内存数据之后,其它对象的内存数据也会同时发生改变。

\vspace{-0.5cm}

\begin{lstlisting}[language=Python]
a = {1:[1, 2, 3]}
b = a
\end{lstlisting}

\begin{figure}[H]
	\centering
	\begin{tikzpicture}
		\draw (0,0) rectangle node{\{1:\textcolor{blue}{[1, 2, 3]}\}} (2.5,1);
		\draw (-2,1.5) circle(0.5) node{a};
		\draw (-2,-0.5) circle(0.5) node{b};
		\draw (3,-2) rectangle node{\textcolor{blue}{[1, 2, 3]}} (5,-1);

		\draw[->] (-1.5,1.5) -- (0,1);
		\draw[->] (-1.5,-0.5) -- (0,0);
		\draw[->, blue] (1.4,0.2) -- (1.4,-1.5) -- (3,-1.5);
	\end{tikzpicture}
	\caption{引用}
\end{figure}

\vspace{0.5cm}

\mybox{引用传递}

\begin{lstlisting}[language=Python]
def main():
    info = dict(name="小灰", age=16, skills=["Python", "C/C++"])
    copy_info = info        # 引用传递
    copy_info["skills"].append("Java")
    print(info)

if __name__ == "__main__":
    main()
\end{lstlisting}

\begin{tcolorbox}
	\mybox{运行结果}
	\begin{verbatim}
{'name':'小灰', 'age':16, 'skills':['Python', 'C/C++', 'Java']}
\end{verbatim}
\end{tcolorbox}

\vspace{0.5cm}

\subsection{浅拷贝}

拷贝和引用传递是不同的,拷贝是将原始的内存的数据进行一份复制,而后为其分配单独的对象的指向。

\vspace{-0.5cm}

\begin{lstlisting}[language=Python]
a = {1:[1, 2, 3]}
b = a.copy()
\end{lstlisting}

\begin{figure}[H]
	\centering
	\begin{tikzpicture}
		\draw (0,1) rectangle node{\{1:\textcolor{blue}{[1, 2, 3]}\}} (2.5,2);
		\draw (-2,1.5) circle(0.5) node{a};
		\draw (0,-1) rectangle node{\{1:\textcolor{red}{[1, 2, 3]}\}} (2.5,0);
		\draw (-2,-0.5) circle(0.5) node{b};
		\draw (4,0) rectangle node{[1, 2, 3]} (6,1);

		\draw[->] (-1.5,1.5) -- (0,1.5);
		\draw[->] (-1.5,-0.5) -- (0,-0.5);
		\draw[->, blue] (1.4,1.2) -- (1.4,0.75) -- (4,0.75);
		\draw[->, red] (1.4,-0.2) -- (1.4,0.3) -- (4,0.3);
	\end{tikzpicture}
	\caption{浅拷贝}
\end{figure}

\vspace{0.5cm}

\mybox{浅拷贝}

\begin{lstlisting}[language=Python]
import copy

def main():
    info = dict(name="小灰", age=16, skills=["Python", "C/C++"])
    copy_info = copy.copy(info)     # 浅拷贝
    copy_info.pop("age")
    copy_info["skills"].append("Java")
    print(info)
    print(copy_info)

if __name__ == "__main__":
    main()
\end{lstlisting}

\begin{tcolorbox}
	\mybox{运行结果}
	\begin{verbatim}
{'name':'小灰', 'age':16, 'skills':['Python', 'C/C++', 'Java']}
{'name':'小灰', 'skills':['Python', 'C/C++', 'Java']}
\end{verbatim}
\end{tcolorbox}

\vspace{0.5cm}

\subsection{深拷贝}

\vspace{-0.5cm}

\begin{lstlisting}[language=Python]
a = {1:[1, 2, 3]}
b = copy.deepcopy(a)
\end{lstlisting}

\begin{figure}[H]
	\centering
	\begin{tikzpicture}
		\draw (0,1) rectangle node{\{1:\textcolor{blue}{[1, 2, 3]}\}} (2.5,2);
		\draw (-2,1.5) circle(0.5) node{a};
		\draw (0,-1) rectangle node{\{1:\textcolor{red}{[1, 2, 3]}\}} (2.5,0);
		\draw (-2,-0.5) circle(0.5) node{b};
		\draw (4,0) rectangle node{\textcolor{blue}{[1, 2, 3]}} (6,1);
		\draw (4,-2) rectangle node{\textcolor{red}{[1, 2, 3]}} (6,-1);

		\draw[->] (-1.5,1.5) -- (0,1.5);
		\draw[->] (-1.5,-0.5) -- (0,-0.5);
		\draw[->, blue] (1.4,1.2) -- (1.4,0.5) -- (4,0.5);
		\draw[->, red] (1.4,-0.8) -- (1.4,-1.5) -- (4,-1.5);
	\end{tikzpicture}
	\caption{深拷贝}
\end{figure}

\vspace{0.5cm}

\mybox{深拷贝}

\begin{lstlisting}[language=Python]
import copy

def main():
    info = dict(name="小灰", age=16, skills=["Python", "C/C++"])
    copy_info = copy.deepcopy(info)     # 深拷贝
    copy_info.pop("age")
    copy_info["skills"].append("Java")
    print(info)
    print(copy_info)

if __name__ == "__main__":
    main()
\end{lstlisting}

\begin{tcolorbox}
	\mybox{运行结果}
	\begin{verbatim}
{'name': '小灰', 'age': 16, 'skills': ['Python', 'C/C++']}
{'name': '小灰', 'skills': ['Python', 'C/C++', 'Java']}
\end{verbatim}
\end{tcolorbox}

\newpage

\section{MapReduce数据处理}

\subsection{MapReduce}

Python在数据分析领域上使用非常广泛,并且实现简单。在Python中可以进行大量数据的快速处理,进行数据的过滤、分析操作。在数据量小的情况下可以方便地使用for循环进行数据的逐个处理,但是在数据量大的情况下,就需要使用一些特定的处理函数进行过滤、分析、统计等操作。\\

在Python中默认提供有了filter()、map(),但是要进行统计处理,则需要导入reduce()。

\begin{table}[H]
	\centering
	\setlength{\tabcolsep}{5mm}{
		\begin{tabular}{|l|l|}
			\hline
			\textbf{函数}              & \textbf{功能}            \\
			\hline
			filter(function, sequence) & 对传入的序列数据进行过滤 \\
			\hline
			map(function, sequence)    & 对传入的序列数据进行处理 \\
			\hline
			reduce(function, sequence) & 对传入的序列数据进行统计 \\
			\hline
		\end{tabular}
	}
	\caption{MapReduce数据处理函数}
\end{table}

在进行数据处理的过程之中都需要有一个处理函数,这个处理函数就定义了数据该如何进行处理或统计,一般而言这样的函数都比较短,所以大部分情况下都可以利用lambda函数来完成。\\

\mybox{MapReduce数据处理}

\begin{lstlisting}[language=Python]
from functools import reduce

def main():
    lst = list(range(10))

    filter_lst = list(filter(lambda x: x % 2 ==0, lst))
    print("过滤出偶数:%s" % filter_lst)

    map_lst = list(map(lambda x: x ** 2, filter_lst))
    print("平方:%s" % map_lst)

    result = reduce(lambda x, y: x+y, map_lst)
    print("求和:%d" % result)

if __name__ == "__main__":
    main()
\end{lstlisting}

\begin{tcolorbox}
	\mybox{运行结果}
	\begin{verbatim}
过滤出偶数:[0, 2, 4, 6, 8]
平方:[0, 4, 16, 36, 64]
求和:120
\end{verbatim}
\end{tcolorbox}

\newpage

\section{pip模块管理工具}

\subsection{pip}

Python本地有一些系统模块开发者可以直接进行使用,但是开发者仅仅是依靠系统模块是不够的,需要大量的使用第三方模块。为了解决这些模块的管理问题,在Python中内置了pip管理工具。通过此工具可以直接连接到Python远程服务模块仓库,通过仓库下载所需要的模块。\\

在Python安装的时候会自动进行pip工具的相关安装,输入pip --help查看pip的相关命令选项。

\begin{table}[H]
	\centering
	\setlength{\tabcolsep}{5mm}{
		\begin{tabular}{|l|l|}
			\hline
			\textbf{功能}  & \textbf{命令}                \\
			\hline
			搜索模块       & pip search 模块名            \\
			\hline
			安装模块       & pip install 模块名           \\
			\hline
			查看已安装模块 & pip list                     \\
			\hline
			列出过期模块   & pip list --outdated          \\
			\hline
			更新模块       & pip install --upgrade 模块名 \\
			\hline
			卸载模块       & pip uninstall 模块名         \\
			\hline
		\end{tabular}
	}
	\caption{pip命令}
\end{table}

\newpage

\section{jieba分词}

\subsection{jieba}

分词是一种数学的应用,它可以直接根据词语之间的数学关系进行文字或单词的抽象。例如对“中华人民共和国”进行分词处理,可以拆分为“中华”、“华人”、“人民”、“共和”、“共和国”、“中华人民共和国”。如果没有分词,就无法进行搜索引擎的开发。\\

jieba是在中文自然语言处理中用得最多的工具包之一,它以分词起家,目前已经能够实现包括分词、词性标注以及命名实体识别等多种功能。\\

考虑到分词的效果与性能,在jieba组件中提供有3种分词模式:

\begin{enumerate}
	\item 精确模式:将句子进行最精确的切分,分词速度相对较低。

	\item 全模式:基于词汇列表将句子中所有可以成词的词语都扫描出来,该模式处理速度非常快,但是不能有效解决歧义的问题。

	\item 搜索引擎模式:在精确模式的基础上,对长词进行再次切分,该模式适用于搜索引擎构建索引的分词。
\end{enumerate}

\vspace{0.5cm}

\mybox{统计《西游记》中出现次数最多的20个词语}

\begin{lstlisting}[language=Python]
import jieba

PATH = "西游记.txt"     # 文件路径

def main():
    word_frequence = {}     # 词频表

    # 打开文件
    with open(file=PATH, mode="r", encoding="UTF-8") as file:
        line = file.readline()  # 读取一行数据
        while line:
            words = jieba.lcut(line)    # 分词
            for word in words:
                if len(word) == 1:  # 舍弃长度为1的词
                    continue
                else:
                    # dict.get(key, default=None)
                    word_frequence[word] 
                        = word_frequence.get(word, 0) + 1
            line = file.readline()

        # 获取所有数据项
        items = list(word_frequence.items())
        # 根据出现次数降序排序
        items.sort(key=lambda x: x[1], reverse=True)

        # 取前20项
        for i in range(20):
            word, count = items[i]
            print("%s: %s" % (word, count))

if __name__ == "__main__":
    main()
\end{lstlisting}

\newpage
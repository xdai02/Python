\chapter{文件操作}

\section{文件操作}

\subsection{文件操作}

计算机对于数据的存储一般可以通过文件的形式来完成。Python中直接提供有文件的I/O(Input/Output)处理函数操作,能够方便地实现读取和写入。\\

open()的功能是进行文件的打开,在进行文件打开的时候如果不设置任何的模式类型,则默认为r(只读模式)。

\vspace{-0.5cm}

\begin{lstlisting}[language=Python]
def open(file, mode='r', buffering=None, encoding=None,
    	errors=None, newline=None, closefd=True
)
\end{lstlisting}

\begin{table}[H]
	\centering
	\setlength{\tabcolsep}{5mm}{
		\begin{tabular}{|c|l|}
			\hline
			\textbf{打开模式} & \textbf{功能}                                  \\
			\hline
			r                 & 使用只读模式打开文件,此为默认模式             \\
			\hline
			w                 & 写模式,如果文件存在则覆盖,文件不存在则创建   \\
			\hline
			x                 & 写模式,新建一个文件,如果该文件已存在则会报错 \\
			\hline
			a                 & 内容追加模式                                   \\
			\hline
			b                 & 二进制模式                                     \\
			\hline
			t                 & 文本模式(默认)                               \\
			\hline
			+                 & 打开一个文件进行更新(可读可写)               \\
			\hline
		\end{tabular}
	}
	\caption{文件打开模式}
\end{table}

如果以只读的模式打开文件,并且文件路径不存在的话,就会出现FileNotFoundError的错误信息。\\

\mybox{文件操作}

\begin{lstlisting}[language=Python]
def main():
	file = open(file="test.txt", mode="w")
	print("文件名称:%s" % file.name)
	print("访问模式:%s" % file.mode)
	print("文件状态:%s" % file.closed)
	print("关闭文件...")
	file.close()
	print("文件状态:%s" % file.closed)

if __name__ == "__main__":
	main()
\end{lstlisting}

\begin{tcolorbox}
	\mybox{运行结果}
	\begin{verbatim}
文件名称:test.txt
访问模式:w
文件状态:False
关闭文件...
文件状态:True
\end{verbatim}
\end{tcolorbox}

\vspace{0.5cm}

\subsection{文件读写}

当使用open()打开一个文件后,就可以使用创建的文件对象进行读写操作。

\begin{table}[H]
	\centering
	\setlength{\tabcolsep}{4mm}{
		\begin{tabular}{|l|l|}
			\hline
			\textbf{方法}                             & \textbf{功能}                  \\
			\hline
			def close(self)                           & 关闭文件资源                   \\
			\hline
			def flush(self)                           & 强制刷新缓冲区                 \\
			\hline
			def read(self, n: int = -1)               & 默认读取全部,也可设置读取个数 \\
			\hline
			def readlines(self, hint: int = -1)       & 读取所有数据行,以列表形式返回 \\
			\hline
			def readline(self, limit: int = -1)       & 读取每行数据,也可设置读取个数 \\
			\hline
			def write(self, s: AnyStr)                & 文件写入                       \\
			\hline
			def writelines(self, lines, List[AnyStr]) & 写入一组数据                   \\
			\hline
		\end{tabular}
	}
	\caption{文件读写}
\end{table}

既然所有的文件对象最终都需要被开发者关闭,那么可以结合with语句实现自动的关闭处理。通过with实现所有资源对象的连接和释放是在Python中编写资源操作的重要技术手段,通过这样的操作可以极大地减少和优化代码结构。\\

使用读模式打开文件后,可以使用循环读取每一行的数据内容。Python在进行文件读取操作的时候也可以进一步简化操作。文件对象本身是可以迭代的,在迭代的时候是以换行符进行分割,每次迭代就读取到一行数据内容。\\

\mybox{读取文件}

\begin{lstlisting}[title=data.txt]
小灰	16
小白	17
小黄	21
\end{lstlisting}

\begin{lstlisting}[language=Python, title=read\_file.py]
def main():
	with open(file="data.txt", mode="r", encoding="utf-8") as file:
		for line in file:
			print(line, end='')

if __name__ == "__main__":
	main()
\end{lstlisting}

\begin{tcolorbox}
	\mybox{运行结果}
	\begin{verbatim}
小灰	16
小白	17
小黄	21
\end{verbatim}
\end{tcolorbox}

\vspace{0.5cm}

\mybox{写入文件}

\begin{lstlisting}[language=Python]
def main():
	with open(file="data.txt", mode="w", encoding="utf-8") as file:
		info = {"小灰": 16, "小白": 17, "小黄": 21}
		for name, age in info.items():
			file.write("%s\t%d\n" % (name, age))

if __name__ == "__main__":
	main()
\end{lstlisting}

\begin{tcolorbox}
	\mybox{运行结果}
	\textbf{data.txt}
	\begin{verbatim}
小灰	16
小白	17
小黄	21
\end{verbatim}
\end{tcolorbox}

\newpage

\section{os模块}

\subsection{os模块}

os模块是Python用于与操作系统进行交互的一个操作模块,这个模块提供有大量与系统相关的处理函数,开发者可以直接通过Python程序进行操作系统的功能调用。

\begin{table}[H]
	\centering
	\setlength{\tabcolsep}{5mm}{
		\begin{tabular}{|l|l|}
			\hline
			\textbf{方法}     & \textbf{功能}      \\
			\hline
			getcwd()          & 获取当前的工作目录 \\
			\hline
			chdir(path)       & 修改工作目录       \\
			\hline
			system()          & 执行操作系统命令   \\
			\hline
			symlink(src, dst) & 创建软链接         \\
			\hline
			link(src, dst)    & 创建硬链接         \\
			\hline
		\end{tabular}
	}
	\caption{os模块}
\end{table}

\vspace{0.5cm}

\subsection{os.path子模块}

os.path是os模块之中的一个子模块,该子模块的核心作用在于进行路径处理操作。Python的程序代码本身是强调跨平台的,既然要进行跨平台的开发,尤其是在I/O路径的处理上就特别要引起注意。在Windows系统下路径分隔符使用的是【$ \backslash $】,而在Linux系统下使用的路径分隔符是【/】。所以在进行程序编写的时候就必须考虑到不同平台的设计问题。\\

使用os.path模块中提供的一系列函数可以针对给定的路径进行拆分处理以及判断和取得数据信息。\\

Python需要考虑不同操作系统的跨平台的特点,所以对于访问路径需要进行适当的变更,根据不同的操作系统使用不同的路径分隔符。如果每一次都判断操作系统就过于繁琐了,可以直接使用os.path中提供的变量来完成。

\begin{table}[H]
	\centering
	\setlength{\tabcolsep}{5mm}{
		\begin{tabular}{|c|l|}
			\hline
			\textbf{变量} & \textbf{功能}                                             \\
			\hline
			curdir        & 表示当前文件夹【.】,一般可以省略                         \\
			\hline
			pardir        & 上一层文件夹【..】                                        \\
			\hline
			sep           & 系统路径分隔符,Windows为【$ \backslash $】,Linux为【/】 \\
			\hline
			extsep        & 文件名称和后缀之间的间隔符【.】                           \\
			\hline
		\end{tabular}
	}
	\caption{路径分隔符}
\end{table}

\vspace{0.5cm}

\mybox{获取路径信息}

\begin{lstlisting}[language=Python]
import os

PATH = "code" + os.sep +"第10章 文件操作" + os.sep \
        + "10.3 os模块" + os.sep + "data.txt"

def main():
    if os.path.exists(PATH):
        print("绝对路径:%s" % os.path.abspath(PATH))
        print("文件名称:%s" % os.path.basename(PATH))
        print("文件大小:%s" % os.path.getsize(PATH))
        print("当前路径是否为文件:%s" % os.path.isfile(PATH))
        print("当前路径是否为目录:%s" % os.path.isdir(PATH))

if __name__ == "__main__":
    main()
\end{lstlisting}

\begin{tcolorbox}
	\mybox{运行结果}
	\begin{verbatim}
绝对路径:C:\Users\Administrator\Desktop\Python\code\第10章 文
件操作\10.3 os模块\data.txt
文件名称:data.txt
文件大小:15
当前路径是否为文件:True
当前路径是否为目录:False
\end{verbatim}
\end{tcolorbox}

\newpage

\section{csv模块}

\subsection{csv文件}

CSV (Comma-Separated Values,逗号分隔值/字符分隔值)是一种文件的格式,在该类型的文件里面一般会保存多个数据信息的内容,但是每一个数据信息一定都有各自的组成部分,用这样的文件进行数据采集内容的记录。CSV文件是跟人工智能和数据分析有直接联系的一种数据存储文件。\\

CSV是一种以纯文件方式进行数据记录的存储格式,在CSV文件内容使用不同的数据行记录数据的内容,每行数据使用特定的符号(一般是逗号)进行数据项的拆分,这样就形成了一种相对简单且通用的数据格式。在实际开发中利用CSV数据格式可以方便实现大数据系统中对于数据采集结果的信息记录,也可以方便进行数据文件的传输,同时CSV文件格式也可以被Excel工具所读取。\\

CSV文件是可以通过Excel工具打开的,当一个CSV文件被创建之后,在Windows系统中会自动和Excel软件进行关联。\\

\subsection{csv读写操作}

在Python中直接提供有csv模块,利用这个模块可以方便地实现数据的写入和读取操作,在CSV文件内容一般对于不同的数据项都要使用逗号分隔。除了数据之外,在CSV文件内容还可以设置文件标题。\\

\mybox{写入csv文件}

\begin{lstlisting}[language=Python]
import csv
import random

HEADER = ["Location", "Longitude", "Latitude"]

def main():
    # 如果不使用newline,那么每行记录之间就会多出一个空行
    with open(file="location.csv", mode="w", 
                newline="", encoding="utf-8") as file:
        csv_writer = csv.writer(file)       # 创建csv写入对象
        csv_writer.writerow(HEADER)         # 写入头部信息
        for i in range(1, 11):
            longitude = round(random.random() * 180, 3)   # [0, 180)
            latitude = round(random.random() * 90, 3)     # [0, 90)
            csv_writer.writerow(["loc-%d" % i, longitude, latitude])

if __name__ == "__main__":
    main()
\end{lstlisting}

\begin{tcolorbox}
	\mybox{运行结果}
	\textbf{location.csv}
	\begin{verbatim}
Location,Longitude,Latitude
loc-1,176.165,35.458
loc-2,12.729,56.247
loc-3,6.605,45.14
loc-4,15.123,53.435
loc-5,131.984,11.927
loc-6,155.038,35.681
loc-7,98.772,15.125
loc-8,70.991,30.328
loc-9,152.967,30.372
loc-10,96.362,76.798
\end{verbatim}
\end{tcolorbox}

\vspace{0.5cm}

\mybox{读取csv文件}

\begin{lstlisting}[language=Python]
import csv

def main():
    with open(file="location.csv", mode="r", 
                newline="", encoding="utf-8") as file:
        csv_reader = csv.reader(file)   # 创建csv读取对象
        header = next(csv_reader)       # 读取标题行
        print(header)
        for row in csv_reader:
            print(row)

if __name__ == "__main__":
    main()
\end{lstlisting}

\begin{tcolorbox}
	\mybox{运行结果}
	\begin{verbatim}
['Location', 'Longitude', 'Latitude']
['loc-1', '176.165', '35.458']
['loc-2', '12.729', '56.247']
['loc-3', '6.605', '45.14']
['loc-4', '15.123', '53.435']
['loc-5', '131.984', '11.927']
['loc-6', '155.038', '35.681']
['loc-7', '98.772', '15.125']
['loc-8', '70.991', '30.328']
['loc-9', '152.967', '30.372']
['loc-10', '96.362', '76.798']
\end{verbatim}
\end{tcolorbox}

\newpage